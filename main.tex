\documentclass[
	12pt,
	openright,
	twoside,
	a4paper,
	english,
	french,
	spanish,
	brazil
]{abntex2}

% ----------------
% Pacotes básicos.
% ----------------
\usepackage{lmodern}
\usepackage[T1]{fontenc}
\usepackage[utf8]{inputenc}
\usepackage{indentfirst}
\usepackage{color}
\usepackage{graphicx}
\usepackage{microtype}

% --------------------
% Pacotes de citações.
% --------------------
\usepackage[brazilian,hyperpageref]{backref}
\usepackage[alf]{abntex2cite}

% -------------------------
% Configurações de pacotes.
% -------------------------

\renewcommand{\backrefpagesname}{Citado na(s) página(s):~}
\renewcommand{\backref}{}
\renewcommand*{\backrefalt}[4]{
	\ifcase #1
		Nenhuma citação no texto.
	\or
		Citado na página #2.
	\else
		Citado #1 vezes nas páginas #2.
	\fi
}

% ------------------------------------------------
% Informações de dados para capa e folha de rosto.
% ------------------------------------------------
\titulo{A Era do Cativeiro Digital e a Ditadura do Comportamento}
\autor{
  Caio Alexandre Ornelas Silva (221007644) \\ Danilo Augusto Ligiero (221007967)
}
\local{Brasília, Brasil}
\data{2023}
\orientador{Vanessa Maria de Castro}
\instituicao{
  Universidade de Brasília - UnB
  \par
  Faculdade UnB Gama - FGA
  \par
  Engenharias
}
\tipotrabalho{Ensaio}

% ---------------------------------------
% Espaçamentos entre linhas e parágrafos.
% ---------------------------------------
\setlength{\parindent}{1.3cm}
\setlength{\parskip}{0.2cm}

% -----------------
% Compila o índice.
% -----------------
\makeindex

% --------------------
% Início do documento.
% --------------------
\begin{document}
  \selectlanguage{brazil}
  \frenchspacing
  \imprimircapa
  \imprimirfolhaderosto*

  % --------
  % Resumos.
  % --------
  \setlength{\absparsep}{18pt}
  \begin{resumo}
    Este ensaio tem o objetivo de expor e colocar em debate a temática
    ``Capitalismo de Vigilância e a Erosão da Privacidade'' que é de extrema
    importância e muitas vezes passa despercebido pela sociedade no geral, mas
    que sempre está presente no cotidiano. Este ensaio terá como principal
    referência teórica a obra de Shoshana Zuboff, ``A Era do Capitalismo de
    Vigilância'', que aborda a respeito do uso da tecnologia para exploração
    inserido em um contexto da vida moderna. Porém, tambem serão utilizados
    grandes pensadores como Montesquieu e Foucault afim de tornar mais completo
    e sólido todo o estudo do debate.

    \textbf{Palavras-chave}:
      Capitalismo de Vigilância, Privacidade, Tecnologia, Sociedade.
  \end{resumo}

  % ------------------
  % Inserir o sumário.
  % ------------------
  % \pdfbookmark[0]{\contentsname}{toc}
  % \tableofcontents*
  % \cleardoublepage

  % -------------------
  % Elementos textuais.
  % -------------------
  \textual

  % -----------
  % Introdução.
  % -----------
  \chapter{Introdução}

  De acordo com a \citeonline{harari-sapiens}, as primeiras grandes tecnologias
  criadas pela humanidade talvez sejam a invenção das ferramentas de pedra (há
  2,5 milhões de anos), a descoberta e aperfeiçoamento do uso do fogo (há 300
  mil anos), além da invenção da agricultura e da construção de barracos para
  moradia (há 12 mil anos). Já na atualidade tem-se a internet, que caminha mais
  recentemente com a inteligência artificial (especificamente IAs generativas),
  como o principal personagem coadjuvante no cenário tecnológico, mas também da
  própria realidade. A internet ganhou tanta força que hoje não se pode viver
  usufruindo da modernidade e urbanização sem o uso constante dela. Partindo
  destes pontos, é mais fácil debater como o viés do uso do poder direcionado ao
  controle massificado (da grande sociedade), através de ferramentas como a
  internet e tecnologias de vigilância, é praticado e a forma como isso afeta a
  sociedade e o indivíduo.

  Nesse contexto, é possível observar que esse processo que se iniciou há cerca
  de 2,8 milhões de anos ocorreu não somente com o objetivo natural intrínseco
  de evolução, mas também com um viés de uso de tais ferramentas para exercer
  poder e controle. No entanto, esse viés não está contido em todo processo,
  provavelmente se inicia de forma mais recente na história na época das
  primeiras civilizações fixas, onde a dinâmica de poder muda completamente e as
  relações políticas entre os indivíduos ganha um palco maior. Isso pode ser
  visto mais fortemente depois do surgimento do capitalismo (século XV-XVI) na
  Europa, mais especificamente após a Terceira Revolução Industrial, onde se
  teve a invenção dos computadores. Shoshana Zuboff mostra em sua obra
  ``Capitalismo de Vigilância'' exatamente como esse vié é aplicado no atual
  contexto capitalista tecnológico, além de fazer uma profunda análise à
  respeito de diversos aspectos contidos nisso.

  A era da internet e suas tecnologias (Terceira Revolução Industrial) podem ser
  descritas como um enorme avanço tecnológico para a humanidade, visto que
  trouxe consigo muitas inovações positivas para a sociedade. Essas inovações
  foram responsáveis por globalizar o mundo e conectar as pessoas através de por
  exemplo plataformas de rede sociais, mas ao mesmo tempo também se tornaram
  grandes ferramentas de controle sistemático e massificado da opinião pública e
  do comportamento da populacao pelas que detém o poder dessas plataformas.

  % ----------------
  % Desenvolvimento.
  % ----------------
  \chapter{A Ideia de Vigilância}

  Um grande pensador que aborda a respeito do uso do poder com fins de controle
  e vigilância é o Michel Foucault. Foucault apresenta seus conceitos que são de
  extrema importância para o entendimento do que é vigilância, através do uso do
  Panóptico, que é um termo concebido pelo filosofo Jeremy Bentham em 1785. Esse
  termo designa a ideia de uma penitenciaria ideal (perfeita) onde um único
  vigilante consegue observar todos os prisioneiros, de forma que estes não
  possam saber se estão ou não sendo observados. O Panóptico faz a inversão
  entre o ato de reprimir e produzir, onde a disciplina é invertida deixando de
  ser punitiva e repressiva e passa a ser modeladora do comportamento de forma
  interna ao indivíduo.


  % -----------------------
  % Elementos pós-textuais.
  % -----------------------
  \postextual

  % ---------------------------
  % Referências bibliográficas.
  % ---------------------------
  \bibliography{refs}
\end{document}
