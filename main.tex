\documentclass[
	12pt,
	openright,
	twoside,
	a4paper,
	english,
	french,
	spanish,
	brazil
]{abntex2}

% ----------------
% Pacotes básicos.
% ----------------
\usepackage{lmodern}
\usepackage[T1]{fontenc}
\usepackage[utf8]{inputenc}
\usepackage{indentfirst}
\usepackage{color}
\usepackage{graphicx}
\usepackage{microtype}

% --------------------
% Pacotes de citações.
% --------------------
\usepackage[brazilian,hyperpageref]{backref}
\usepackage[alf]{abntex2cite}

% -------------------------
% Configurações de pacotes.
% -------------------------

\renewcommand{\backrefpagesname}{Citado na(s) página(s):~}
\renewcommand{\backref}{}
\renewcommand*{\backrefalt}[4]{
	\ifcase #1
		Nenhuma citação no texto.
	\or
		Citado na página #2.
	\else
		Citado #1 vezes nas páginas #2.
	\fi
}

% ------------------------------------------------
% Informações de dados para capa e folha de rosto.
% ------------------------------------------------
\titulo{A Era do Cativeiro Digital e a Ditadura do Comportamento}
\autor{
  Caio Alexandre Ornelas Silva (221007644) \\ Danilo Augusto Ligiero (221007967)
}
\local{Brasília, Brasil}
\data{2023}
\orientador{Vanessa Maria de Castro}
\instituicao{
  Universidade de Brasília - UnB
  \par
  Faculdade UnB Gama - FGA
  \par
  Engenharias
}
\tipotrabalho{Ensaio}

% ---------------------------------------
% Espaçamentos entre linhas e parágrafos.
% ---------------------------------------
\setlength{\parindent}{1.3cm}
\setlength{\parskip}{0.2cm}

% -----------------
% Compila o índice.
% -----------------
\makeindex

% --------------------
% Início do documento.
% --------------------
\begin{document}
  \selectlanguage{brazil}
  \frenchspacing
  \imprimircapa
  \imprimirfolhaderosto*

  % --------
  % Resumos.
  % --------
  \setlength{\absparsep}{18pt}
  \begin{resumo}
    Este ensaio tem o objetivo de expor e colocar em debate a temática
    ``Capitalismo de Vigilância e a Erosão da Privacidade'' que é de extrema
    importância e muitas vezes passa despercebido pela sociedade no geral, mas
    que sempre está presente no cotidiano. Este ensaio terá como principal
    referência teórica a obra de Shoshana Zuboff, ``A Era do Capitalismo de
    Vigilância'', que aborda a respeito do uso da tecnologia para exploração
    inserido em um contexto da vida moderna. Porém, tambem serão utilizados
    grandes pensadores como Montesquieu e Foucault afim de tornar mais completo
    e sólido todo o estudo do debate.

    \textbf{Palavras-chave}:
      Capitalismo de Vigilância, Privacidade, Tecnologia, Sociedade.
  \end{resumo}

  % ------------------
  % Inserir o sumário.
  % ------------------
  % \pdfbookmark[0]{\contentsname}{toc}
  % \tableofcontents*
  % \cleardoublepage

  % -------------------
  % Elementos textuais.
  % -------------------
  \textual

  % -----------
  % Introdução.
  % -----------
  \chapter{Introdução}

  De acordo com a \citeauthoronline{harari-sapiens}, as primeiras grandes
  tecnologias criadas pela humanidade talvez sejam a invenção das ferramentas de
  pedra (há 2,5 milhões de anos), a descoberta e aperfeiçoamento do uso do fogo
  (há 300 mil anos), além da invenção da agricultura e da construção de barracos
  para moradia (há 12 mil anos) \cite{harari-sapiens}. Já na atualidade tem-se a
  internet, que caminha mais recentemente com a inteligência artificial
  (especificamente IAs generativas), como o principal personagem coadjuvante no
  cenário tecnológico, mas também da própria realidade. A internet ganhou tanta
  força que hoje não se pode viver usufruindo da modernidade e urbanização sem o
  uso constante dela. Partindo destes pontos, é mais fácil debater como o viés
  do uso do poder direcionado ao controle massificado (da grande sociedade),
  através de ferramentas como a internet e tecnologias de vigilância, é
  praticado e a forma como isso afeta a sociedade e o indivíduo.

  Nesse contexto, é possível observar que esse processo que se iniciou há cerca
  de 2,5 milhões de anos ocorreu não somente com o objetivo natural intrínseco
  de evolução, mas também com um viés de uso de tais ferramentas para exercer
  poder e controle. No entanto, esse viés não está contido em todo processo,
  provavelmente se inicia de forma mais recente na história na época das
  primeiras civilizações fixas, onde a dinâmica de poder muda completamente e as
  relações políticas entre os indivíduos ganha um palco maior. Isso pode ser
  visto mais fortemente depois do surgimento do capitalismo (século XV-XVI) na
  Europa, mais especificamente após a Terceira Revolução Industrial, onde se
  teve a invenção dos computadores.
  \citeauthoronline{zuboff-capitalismo-vigilancia} mostra em sua obra
  ``Capitalismo de Vigilância'' exatamente como esse viés é aplicado no atual
  contexto capitalista tecnológico, além de fazer uma profunda análise à
  respeito de diversos aspectos contidos nisso.

  A era da internet e suas tecnologias (Terceira Revolução Industrial) podem ser
  descritas como um enorme avanço tecnológico para a humanidade, visto que
  trouxe consigo muitas inovações positivas para a sociedade. Essas inovações
  foram responsáveis por globalizar o mundo e conectar as pessoas através de,
  por exemplo, plataformas de rede sociais, mas ao mesmo tempo também se
  tornaram grandes ferramentas de controle sistemático e massificado da opinião
  pública e do comportamento da população pelas que detém o poder dessas
  plataformas.

  % ----------------
  % Desenvolvimento.
  % ----------------
  \chapter{O Cativeiro Digital}

  \section{A Ideia de Vigilância}

  Um grande pensador que aborda a respeito do uso do poder com fins de controle
  e vigilância é \citeauthoronline{foucault-vigiar-punir}. Foucault apresenta
  seus conceitos que são de extrema importância para o entendimento do que é
  vigilância, através do uso do Panóptico, que é um termo concebido pelo
  filósofo \citeauthoronline{bentham-panoptico} em 1785
  \cite{bentham-panoptico}. Esse termo designa a ideia de uma penitenciária
  vigilante consegue observar todos os prisioneiros, de forma que estes não
  ideal (perfeita) onde um único possam saber se estão ou não sendo observados.
  O Panóptico faz a inversão entre o ato de reprimir e produzir, onde a
  disciplina é invertida deixando de ser punitiva e repressiva e passa a ser
  modeladora do comportamento de forma interna ao indivíduo
  \cite{trindade-foucault-panoptico}.

  Partindo de um ponto de vista histórico, é observável que o uso do poder no
  geral iniciou-se pela aplicação direta da força, que naturalmente para o ser
  humano era sua principal fonte de poder de forma mais imediata. Porém, com o
  passar do tempo o ser humano como espécie foi desenvolvendo sua faculdade
  cognitiva e passou a utilizar proporcionalmente menos seu poder físico e mais
  sua capacidade de pensar, ou seja, deixando de exercer poder externamente e
  passando para o interno. Ao invés de controlar grandes grupos de pessoas
  através da força física, reprimindo e punindo fisicamente (como, por exemplo,
  a escravidão de forma geral), atualmente os indivíduos em posição de poder
  perceberam que é mais efetivo controlar a mente das massas. Isso tem reflexo
  direto no uso constante de ferramentas digitais no mundo moderno.

  De acordo com as análises de Foucault, ele destaca as práticas punitivas ao
  longo do tempo, desde os suplícios públicos até a ascensão da prisão como
  instrumento central de controle \cite{foucault-vigiar-punir}. A análise
  detalhada dessas mudanças revela como as instituições moldam não apenas o
  corpo, mas também as mentes dos indivíduos. Nesse sentido, é possível combinar
  as ideias de \citeauthoronline{zuboff-capitalismo-vigilancia} com a ideia de
  vigilância de Foucault, e a compreensão dessa mescla se mostra relevante ao
  examinarmos as dinâmicas das redes sociais e tecnologias de conexão
  massificada, onde a vigilância assume formas digitais, mas não menos e sim
  exponencialmente mais impactantes e eficazes.


  \section{Controle de Narrativas}

  O livro ``1984'', escrito por \citeauthoronline{orwell-1984}, é uma obra de
  ficção distópica que se passa em um futuro autoritário. Neste futuro é criado
  um ``Ministério da Verdade'', que trabalha para alterar registros históricos
  para se adequar à narrativa do Partido \cite{orwell-1984}. Por mais que a obra
  se trate de uma realidade fictícia, é possível traçarmos um paralelo com a
  sociedade atual, onde o autoritarismo praticado pelo Partido agora é feito por
  empresas \textit{big techs} de acordo com seus interesses.

  Esse controle autoritário se dá devido ao poder que essas empresas têm sobre o
  conteúdo que é inserido dentro das redes sociais através de complexos
  algoritmos e inteligências artificiais. Elas podem controlar que tipo de
  opinião pública será engajada ou não ou como anúncios serão exibidos para o
  usuário, podendo ser extremamente invasivos, como visto pelo escândalo de
  dados da Cambridge Analytica em 2018 \cite{chan-cambridge-analytica}. Esse é
  um exemplo de como empresas exercem um controle quase que imperceptível em
  seus usuários para manipular seus pensamentos e comportamentos, moldando uma
  realidade digital sob medida para seus interesses comerciais e isso seria a
  personalização digital.

  Semelhante ao Ministério da Verdade em ``1984'', as \textit{big techs} assumem
  o papel de \textit{gatekeepers} da informação, decidindo quais narrativas são
  amplificadas e quais são suprimidas. Esse controle sobre a verdade e a
  narrativa não apenas reflete, mas também influencia a opinião pública, gerando
  preocupações sobre a manipulação da democracia e a liberdade de expressão. O
  paralelo entre a distopia de \citeauthoronline{orwell-1984} e a realidade
  contemporânea é evidente na vigilância constante exercida por empresas que
  coletam e utilizam dados pessoais de maneira ubíqua e não transparente. A obra
  alerta para o perigo de um Estado totalitário que controla a realidade e hoje
  vemos entidades não governamentais exercendo um poder igualmente impactante,
  se não mais.

  Além disso, a noção de Novilíngua em ``1984'', uma linguagem manipulada para
  limitar o pensamento crítico, encontra eco na forma como o discurso online é
  moldado pelas plataformas digitais. Palavras são censuradas, algoritmos
  decidem o que é aceitável, e a diversidade de pensamento muitas vezes é
  sufocada em prol da conformidade com as diretrizes impostas.

  Em última análise, ``1984'' permanece relevante, pois serve como um aviso
  sobre os perigos do controle totalitário e da manipulação da informação. Ao
  observar a interseção entre a ficção distópica e a realidade contemporânea,
  somos instigados a refletir criticamente sobre os desafios éticos e sociais
  que enfrentamos em uma era digital dominada por algoritmos e interesses
  corporativos.


  \section{O Uso do Poder}

  \citeauthoronline{maquiavel-principe}, em ``O Príncipe'', aborda a natureza do
  poder e as estratégias para mantê-lo. Ele argumenta que o governante eficaz
  deve ser astuto e adaptável, utilizando meios que se adequem às circunstâncias
  para manter o controle e a estabilidade em seu principado. Há uma ênfase na
  necessidade de manter o poder de forma eficaz, muitas vezes por meio de
  estratégias que podem ser consideradas cruéis ou pragmáticas
  \cite{maquiavel-principe}. Nesse sentido, seguindo essa logica do uso do
  poder, atualmente os indivíduos em posições de poder (e essas posições são
  variadas) seguem o mesmo pensamento da época do príncipe Lourenço de Médici,
  no sentido de utilizar técnicas e ferramentas sistemáticas para permanecer em
  tais posições. \citeauthoronline{zuboff-capitalismo-vigilancia} evidencia isso
  relacionando o contexto da evolução das altas tecnologias (maior enfoque na
  internet) justamente com a manutenção do \textit{status quo} e monitoramento
  massificado.

  Ao relacionar essas ideias com a dinâmica da era do cativeiro digital,
  percebemos paralelos na forma como as empresas de tecnologia e plataformas de
  mídia social empregam estratégias de controle e influência.
  \citeauthoronline{maquiavel-principe} destacaria a importância de conhecer
  profundamente aqueles que estão sob seu domínio, uma habilidade que as
  \textit{big techs} demonstram ao coletar e analisar dados massivos sobre os
  usuários. Esses dados massivos sao chamados de \textit{big data}, que de forma
  resumida é a reunião de diversas outras informações menores (a
  \textit{small data}) de cada indivíduo e elas são extraídas em transações
  econômicas, internet das coisas, dados governamentais e corporativos, câmeras
  de vigilâncias públicas e privadas. Essas \textit{small datas} e
  \textit{big datas} são agregados, analisados, manipulados, embalados e
  posteriormente vendidos, e isso é denominado como \textit{data exhaust}. A
  ditadura do comportamento também pode ser relacionada à visão maquiavélica de
  que, em alguns casos, a manutenção do poder pode exigir medidas rigorosas. No
  universo digital, algoritmos são ajustados para moldar o comportamento dos
  usuários, seja através de sugestões personalizadas de conteúdo, seja por meio
  de anúncios direcionados. Essas práticas refletem a busca pela manutenção do
  controle sobre a atenção e o engajamento dos usuários.

  Outro importante conceito introduzido pela
  \citeauthoronline{zuboff-capitalismo-vigilancia} é a chamada ``textualização
  eletrônica''. Em suma, este conceito explica como que cada vez mais as ações e
  até os pensamentos das pessoas sao mediados por computadores
  \cite{zuboff-capitalismo-vigilancia}. Com o constante aumento do uso e
  consequentemente da dependência das máquinas para a realização de tarefas
  cotidianamente simples, o ser humano cada vez mais precisa traduzir e
  estruturar todas as suas ações para fazer que o computador entenda o que é
  pedido.

  Um exemplo atual é o algoritmo da seção \textit{For You} do Twitter, que ficou
  código aberto em 2023 \cite{github-for-you}. Fazendo uma análise estática do
  código (ou seja, sem executá-lo), é possível notar que este algoritmo coleta
  dados de métricas relacionados ao posicionamento político do usuário, como por
  exemplo, se ele é um ``democrata'' ou ``republicano''. Esses dados são
  coletados como métricas pelo Twitter, porém o que é feito com esses dados não
  é transparente com os usuários.

  Assim como Maquiavel advogava por uma abordagem pragmática e adaptável para
  governar, as empresas na era digital frequentemente ajustam suas estratégias
  de controle com base nas mudanças nas tendências de comportamento online. A
  análise e manipulação de dados, à luz das ideias maquiavélicas, tornam-se
  ferramentas essenciais na dinâmica do cativeiro digital. A constante adaptação
  do que é exposto, vendido, discutido e problematizado nas mídias é uma forte
  característica do funcionamento do capitalismo atualmente. Dessa forma, a
  mercadoria da vez são as \textit{small datas} as quais estão em constante
  mudança e adaptação, assim para as empresas conseguirem textualizar as ações
  humanas controlando com eficácia o comportamento e o pensamento humano, é
  preciso que elas também se adaptem modelando suas ações para sistematizar e
  massificar cada vez mais de forma constante com o objetivo de controlar e
  vigiar a população \cite{zuboff-capitalismo-vigilancia}.

  Atualmente, como foi dito anteriormente, a mercadoria ``do momento'' é a
  informação do indivíduo. A transformação dessas informações em mercadoria
  útil que pode ser disponibilizada para venda e para exercer poder através do
  controle, é o chamado ``\textit{Capitalismo de vigilância}''. É um novo
  momento no modo de produção capitalista onde a informação sobre a ação humana
  (ou textualização da ação humana) tem um papel importante para o
  desenvolvimento do valor sobre as coisas. Prever o que as pessoas vão querer
  ou até fazê-las quererem coisas específicas.

  \begin{citacao}
    Essa nova forma de capitalismo de informação procura prever e modificar o
    comportamento humano como meio de produzir receitas e controle de mercado
    \cite{zuboff-capitalismo-vigilancia}
  \end{citacao}

  O ``Capitalismo de Vigilância'' de
  \citeauthoronline{zuboff-capitalismo-vigilancia} oferece uma perspectiva
  inovadora sobre o uso do poder na era digital, complementando as reflexões de
  \citeauthoronline{orwell-1984} em seu livro ``1984''. Em seu livro, Zuboff
  explora como as grandes corporações, as chamadas \textit{big techs}, utilizam
  a coleta massiva de dados para moldar o comportamento humano e maximizar seus
  lucros \cite{zuboff-capitalismo-vigilancia}. Este fenômeno revela uma nova
  forma de poder que transcende fronteiras físicas, operando nos cantos mais
  íntimos do ciberespaço. Assim como em ``1984'', onde o Estado exerce um
  controle absoluto sobre a realidade e a verdade, o capitalismo de vigilância
  confere às empresas tecnológicas um poder extraordinário sobre a informação e
  as interações online. As \textit{big techs} tornam-se arquitetas da realidade
  digital, influenciando a percepção e as escolhas dos usuários por meio da
  personalização algorítmica. Ao traçar um paralelo entre essas obras, podemos
  destacar como ambas alertam sobre os riscos do poder desmedido. Enquanto em
  ``1984'' o Estado busca controlar o pensamento e a história, no capitalismo
  de vigilância, as corporações buscam antecipar e moldar desejos e
  comportamentos, criando uma forma mais sutil, mas igualmente eficaz, de
  manipulação.


  \section{Poder na Era Moderna}

  \citeauthoronline{montesquieu-espirito-leis}, em sua obra ``O Espírito das
  Leis'', defende a ideia da separação dos poderes como um princípio fundamental
  para evitar abusos de poder e garantir a liberdade. Ele propõe que o poder
  deve ser dividido entre o legislativo, executivo e judiciário, cada um
  exercendo suas funções de maneira independente e equilibrada
  \cite{montesquieu-espirito-leis}.

  Ao analisar o ``1984'' de \citeauthoronline{orwell-1984} e o ``Capitalismo de
  Vigilância'' de \citeauthoronline{zuboff-capitalismo-vigilancia}, percebemos
  uma convergência no alerta para os perigos de concentração de poder.
  Montesquieu destacaria a importância de impedir que o controle totalitário,
  seja pelo Estado como em ``1984" ou por grandes corporações como no
  capitalismo de vigilância, seja centralizado em uma única entidade. A
  concentração de poder, conforme Montesquieu argumentaria, poderia levar a
  abusos e à supressão da liberdade. No entanto, as obras de Orwell e Zuboff
  também mostram desafios específicos da era contemporânea que Montesquieu não
  pôde prever. Enquanto Montesquieu concentrou-se na separação dos poderes
  governamentais, as preocupações modernas envolvem entidades não
  governamentais, como as \textit{big techs}, que exercem um poder significativo
  sobre a informação e o comportamento, sem necessariamente se enquadrar nas
  divisões clássicas dos poderes.

  Assim, ao relacionar esses textos, podemos perceber que, embora Montesquieu
  tenha estabelecido uma base crucial para a compreensão do equilíbrio de poder,
  as dinâmicas contemporâneas demandam uma adaptação desses princípios para
  lidar com as novas formas de influência e controle, tanto pelo Estado quanto
  por entidades privadas. A reflexão sobre a separação de poderes, então, não
  se limita apenas à esfera governamental, mas se estende a estruturas sociais e
  tecnológicas que moldam nossa realidade. O controle do poder nas mãos das
  \textit{big techs} é evidenciado pela sua capacidade de moldar o que é visto,
  ouvido e compartilhado online. Através da análise algorítmica dos dados
  coletados, essas empresas podem prever e influenciar decisões, seja na esfera
  do consumo ou nas escolhas políticas. O poder, assim como descrito por Zuboff,
  é exercido não apenas para obter lucro, mas para estabelecer uma hegemonia
  digital que permeia todos os aspectos da vida contemporânea
  \cite{zuboff-capitalismo-vigilancia}.

  No entanto, assim como a resistência de Winston Smith em ``1984'', a
  conscientização sobre o capitalismo de vigilância pode abrir caminho para
  questionamentos e mudanças. Compreender o uso do poder nessas narrativas
  paralelas oferece uma visão crítica sobre os desafios éticos e sociais que
  enfrentamos na era digital. As reflexões de Zuboff e Orwell convergem,
  alertando sobre os perigos de um poder descontrolado, seja nas mãos de um
  Estado totalitário ou de conglomerados corporativos, instigando a sociedade a
  repensar as dinâmicas de poder que moldam nosso presente e futuro.

  % ----------
  % Conclusão.
  % ----------
  \chapter{Conclusão}

  Em conclusão, a análise dos textos anteriores, que abordaram temas como
  controle social, poder na era digital, e a influência de algoritmos, oferece
  uma visão multifacetada sobre os desafios éticos e sociais que enfrentamos na
  sociedade contemporânea. A interseção das ideias de
  \citeauthoronline{montesquieu-espirito-leis}, \citeauthoronline{orwell-1984},
  \citeauthoronline{zuboff-capitalismo-vigilancia} e
  \citeauthoronline{maquiavel-principe} destaca a relevância desses pensadores
  para compreendermos as complexidades da vigilância digital, a manipulação da
  informação e a dinâmica de controle em diversas esferas.

  A obra de \citeauthoronline{zuboff-capitalismo-vigilancia}, ao explorar o
  capitalismo de vigilância, acrescenta uma camada crucial à discussão,
  revelando como as corporações, na busca incessante por dados, desempenham um
  papel central no controle social. A autora explica e destaca diversos
  conceitos importantes e triviais para o entendimento do como, porque e a
  relevância desse debate. É perceptível após a leitura das obras citadas, em
  especial a obra sobre vigilância digital de
  \citeauthoronline{zuboff-capitalismo-vigilancia}, o que acontece atualmente no
  mundo quando se fala a respeito de controle social. Além disso, ela propõe
  ideias extremamente complexas e profundas na forma de funcionamento das
  interações sociais que são diretamente ditadas pela economia e pelos
  controladores, apresentando, por exemplo, o conceito de capitalismo de
  vigilância.

  Em última análise, a convergência dessas ideias ressalta a necessidade
  contínua de reflexão crítica e responsabilidade ética na era digital. O
  desafio é encontrar um equilíbrio entre a conveniência proporcionada pela
  personalização digital e a preservação dos princípios fundamentais de
  liberdade, transparência e separação de poderes. A compreensão desses dilemas
  éticos é essencial para forjar um caminho que harmonize a inovação tecnológica
  com os valores que fundamentam uma sociedade justa e livre.

  % -----------------------
  % Elementos pós-textuais.
  % -----------------------
  \postextual

  % ---------------------------
  % Referências bibliográficas.
  % ---------------------------
  \bibliography{refs}
\end{document}
