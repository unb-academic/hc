\documentclass[
	12pt,
	openright,
	twoside,
	a4paper,
	english,
	french,
	spanish,
	brazil
]{abntex2}

% ----------------
% Pacotes básicos.
% ----------------
\usepackage{lmodern}
\usepackage[T1]{fontenc}
\usepackage[utf8]{inputenc}
\usepackage{indentfirst}
\usepackage{color}
\usepackage{graphicx}
\usepackage{microtype}

% --------------------
% Pacotes de citações.
% --------------------
\usepackage[brazilian,hyperpageref]{backref}
\usepackage[alf]{abntex2cite}

% -------------------------
% Configurações de pacotes.
% -------------------------

\renewcommand{\backrefpagesname}{Citado na(s) página(s):~}
\renewcommand{\backref}{}
\renewcommand*{\backrefalt}[4]{
	\ifcase #1
		Nenhuma citação no texto.
	\or
		Citado na página #2.
	\else
		Citado #1 vezes nas páginas #2.
	\fi
}

% ------------------------------------------------
% Informações de dados para capa e folha de rosto.
% ------------------------------------------------
\titulo{A Era do Cativeiro Digital e a Ditadura do Comportamento}
\autor{
  Caio Alexandre Ornelas Silva (221007644) \\ Danilo Augusto Ligiero (221007967)
}
\local{Brasília, Brasil}
\data{2023}
\orientador{Vanessa Maria de Castro}
\instituicao{
  Universidade de Brasília - UnB
  \par
  Faculdade UnB Gama - FGA
  \par
  Engenharias
}
\tipotrabalho{Ensaio}

% ---------------------------------------
% Espaçamentos entre linhas e parágrafos.
% ---------------------------------------
\setlength{\parindent}{1.3cm}
\setlength{\parskip}{0.2cm}

% -----------------
% Compila o índice.
% -----------------
\makeindex

% --------------------
% Início do documento.
% --------------------
\begin{document}
  \selectlanguage{brazil}
  \frenchspacing
  \imprimircapa
  \imprimirfolhaderosto*

  % --------
  % Resumos.
  % --------
  \setlength{\absparsep}{18pt}
  \begin{resumo}
    Este ensaio tem o objetivo de expor e colocar em debate a temática
    ``Capitalismo de Vigilância e a Erosão da Privacidade'' que é de extrema
    importância e muitas vezes passa despercebido pela sociedade no geral, mas
    que sempre está presente no cotidiano. Este ensaio terá como principal
    referência teórica a obra de Shoshana Zuboff, ``A Era do Capitalismo de
    Vigilância'', que aborda a respeito do uso da tecnologia para exploração
    inserido em um contexto da vida moderna. Porém, tambem serão utilizados
    grandes pensadores como Montesquieu e Foucault afim de tornar mais completo
    e sólido todo o estudo do debate.

    \textbf{Palavras-chave}:
      Capitalismo de Vigilância, Privacidade, Tecnologia, Sociedade.
  \end{resumo}

  % ------------------
  % Inserir o sumário.
  % ------------------
  % \pdfbookmark[0]{\contentsname}{toc}
  % \tableofcontents*
  % \cleardoublepage

  % -------------------
  % Elementos textuais.
  % -------------------
  \textual

  % -----------
  % Introdução.
  % -----------
  \chapter{Introdução}

  De acordo com a \citeauthoronline{harari-sapiens}, as primeiras grandes
  tecnologias criadas pela humanidade talvez sejam a invenção das ferramentas de
  pedra (há 2,5 milhões de anos), a descoberta e aperfeiçoamento do uso do fogo
  (há 300 mil anos), além da invenção da agricultura e da construção de barracos
  para moradia (há 12 mil anos) \cite{harari-sapiens}. Já na atualidade tem-se a
  internet, que caminha mais recentemente com a inteligência artificial
  (especificamente IAs generativas), como o principal personagem coadjuvante no
  cenário tecnológico, mas também da própria realidade. A internet ganhou tanta
  força que hoje não se pode viver usufruindo da modernidade e urbanização sem o
  uso constante dela. Partindo destes pontos, é mais fácil debater como o viés
  do uso do poder direcionado ao controle massificado (da grande sociedade),
  através de ferramentas como a internet e tecnologias de vigilância, é
  praticado e a forma como isso afeta a sociedade e o indivíduo.

  Nesse contexto, é possível observar que esse processo que se iniciou há cerca
  de 2,5 milhões de anos ocorreu não somente com o objetivo natural intrínseco
  de evolução, mas também com um viés de uso de tais ferramentas para exercer
  poder e controle. No entanto, esse viés não está contido em todo processo,
  provavelmente se inicia de forma mais recente na história na época das
  primeiras civilizações fixas, onde a dinâmica de poder muda completamente e as
  relações políticas entre os indivíduos ganha um palco maior. Isso pode ser
  visto mais fortemente depois do surgimento do capitalismo (século XV-XVI) na
  Europa, mais especificamente após a Terceira Revolução Industrial, onde se
  teve a invenção dos computadores. Shoshana Zuboff mostra em sua obra
  ``Capitalismo de Vigilância'' exatamente como esse viés é aplicado no atual
  contexto capitalista tecnológico, além de fazer uma profunda análise à
  respeito de diversos aspectos contidos nisso.

  A era da internet e suas tecnologias (Terceira Revolução Industrial) podem ser
  descritas como um enorme avanço tecnológico para a humanidade, visto que
  trouxe consigo muitas inovações positivas para a sociedade. Essas inovações
  foram responsáveis por globalizar o mundo e conectar as pessoas através de,
  por exemplo, plataformas de rede sociais, mas ao mesmo tempo também se
  tornaram grandes ferramentas de controle sistemático e massificado da opinião
  pública e do comportamento da população pelas que detém o poder dessas
  plataformas.

  % ----------------
  % Desenvolvimento.
  % ----------------
  \chapter{A Ideia de Vigilância}

  Um grande pensador que aborda a respeito do uso do poder com fins de controle
  e vigilância é \citeauthoronline{foucault-vigiar-punir}. Foucault apresenta
  seus conceitos que são de extrema importância para o entendimento do que é
  vigilância, através do uso do Panóptico, que é um termo concebido pelo
  filósofo \citeauthoronline{bentham-panoptico} em 1785
  \cite{bentham-panoptico}. Esse termo designa a ideia de uma penitenciária
  vigilante consegue observar todos os prisioneiros, de forma que estes não
  ideal (perfeita) onde um único possam saber se estão ou não sendo observados.
  O Panóptico faz a inversão entre o ato de reprimir e produzir, onde a
  disciplina é invertida deixando de ser punitiva e repressiva e passa a ser
  modeladora do comportamento de forma interna ao indivíduo
  \cite{trindade-foucault-panoptico}.

  Partindo de um ponto de vista histórico, é observável que o uso do poder no
  geral iniciou-se pela aplicação direta da força, que naturalmente para o ser
  humano era sua principal fonte de poder de forma mais imediata. Porém, com o
  passar do tempo o ser humano como espécie foi desenvolvendo sua faculdade
  cognitiva e passou a utilizar proporcionalmente menos seu poder físico e mais
  sua capacidade de pensar, ou seja, deixando de exercer poder externamente e
  passando para o interno. Ao invés de controlar grandes grupos de pessoas
  através da força física, reprimindo e punindo fisicamente (como, por exemplo,
  a escravidão de forma geral), atualmente os indivíduos em posição de poder
  perceberam que é mais efetivo controlar a mente das massas. Isso tem reflexo
  direto no uso constante de ferramentas digitais no mundo moderno.

  De acordo com as análises de Foucault, ele destaca as práticas punitivas ao
  longo do tempo, desde os suplícios públicos até a ascensão da prisão como
  instrumento central de controle \cite{foucault-vigiar-punir}. A análise
  detalhada dessas mudanças revela como as instituições moldam não apenas o
  corpo, mas também as mentes dos indivíduos. Nesse sentido, é possível combinar
  as ideias de Shoshana Zuboff com a ideia de vigilância de Foucault, e essa
  compreensão dessa mescla se mostra relevante ao examinarmos as dinâmicas das
  redes sociais e tecnologias de conexão massificada, onde a vigilância assume
  formas digitais, mas não menos e sim exponencialmente mais impactantes e
  eficazes.


  \chapter{Controle de Narrativas}

  O livro ``1984'', escrito por \citeauthoronline{orwell-1984}, é uma obra de
  ficção distópica que se passa em um futuro autoritário. Neste futuro é criado
  um ``Ministério da Verdade'', que trabalha para alterar registros históricos
  para se adequar à narrativa do Partido \cite{orwell-1984}. Por mais que a obra
  se trate de uma realidade fictícia, é possível traçarmos um paralelo com a
  sociedade atual, onde o autoritarismo praticado pelo Partido agora é feito por
  empresas \textit{big techs} de acordo com seus interesses.

  Esse controle autoritário se dá devido ao poder que essas empresas têm sobre o
  conteúdo que é inserido dentro das redes sociais através de complexos
  algoritmos e inteligências artificiais. Elas podem controlar que tipo de
  opinião pública será engajada ou não ou como anúncios serão exibidos para o
  usuário, podendo ser extremamente invasivos, como visto pelo escândalo de
  dados da Cambridge Analytica em 2018 \cite{chan-cambridge-analytica}. Esse é
  um exemplo de como empresas exercem um controle quase que imperceptível em
  seus usuários para manipular seus pensamentos e comportamentos, moldando uma
  realidade digital sob medida para seus interesses comerciais e isso seria a
  personalização digital.

  Semelhante ao Ministério da Verdade em ``1984'', as \textit{big techs} assumem
  o papel de \textit{gatekeepers} da informação, decidindo quais narrativas são
  amplificadas e quais são suprimidas. Esse controle sobre a verdade e a
  narrativa não apenas reflete, mas também influencia a opinião pública, gerando
  preocupações sobre a manipulação da democracia e a liberdade de expressão. O
  paralelo entre a distopia de \citeauthoronline{orwell-1984} e a realidade
  contemporânea é evidente na vigilância constante exercida por empresas que
  coletam e utilizam dados pessoais de maneira ubíqua e não transparente. A obra
  alerta para o perigo de um Estado totalitário que controla a realidade, e
  hoje, vemos entidades não governamentais exercendo um poder igualmente
  impactante, se não mais.

  Além disso, a noção de Novilíngua em ``1984'', uma linguagem manipulada para
  limitar o pensamento crítico, encontra eco na forma como o discurso online é
  moldado pelas plataformas digitais. Palavras são censuradas, algoritmos
  decidem o que é aceitável, e a diversidade de pensamento muitas vezes é
  sufocada em prol da conformidade com as diretrizes impostas.

  Em última análise, ``1984'' permanece relevante, pois serve como um aviso sobre
  os perigos do controle totalitário e da manipulação da informação. Ao observar
  a interseção entre a ficção distópica e a realidade contemporânea, somos
  instigados a refletir criticamente sobre os desafios éticos e sociais que
  enfrentamos em uma era digital dominada por algoritmos e interesses
  corporativos.

  % -----------------------
  % Elementos pós-textuais.
  % -----------------------
  \postextual

  % ---------------------------
  % Referências bibliográficas.
  % ---------------------------
  \bibliography{refs}
\end{document}
